\documentclass{article}
\pagestyle{plain}
\usepackage[utf8]{inputenc}
\usepackage{tabularx}
\usepackage{amsmath,amssymb,amsfonts}
\usepackage{algorithmic}
\usepackage{graphicx}
\usepackage{xcolor}
\usepackage[skip=-2pt]{subcaption}
\usepackage[skip=5pt]{caption}
\usepackage[pdftex,
            pdfauthor=Jimmy Aguilar Mena,
            pdftitle={OmpSs-2@Cluster: Distributed memory execution of nested OpenMP-style tasks},
            pdfsubject={Europar-2022},
            pdfkeywords={OmpSs-2,Task,Cluster},
            pdfproducer={Latex with hyperref},
            pdfcreator={pdflatex}]{hyperref}
\usepackage{booktabs}

\usepackage{listings}
\lstset{numbers=left,basicstyle=\scriptsize,language=bash}



\let\origthelstnumber\thelstnumber

\newcommand*\Suppressnumber{%
  \lst@AddToHook{OnNewLine}{%
    \let\thelstnumber\relax%
     \advance\c@lstnumber-\@ne\relax%
    }%
}
\newcommand*\Reactivatenumber{%
  \lst@AddToHook{OnNewLine}{%
   \let\thelstnumber\origthelstnumber%
   \advance\c@lstnumber\@ne\relax}%
}

\usepackage{sansmath}

\newcommand{\prag}[1]{\textcolor{blue}{#1}}

\usepackage[T1]{fontenc}
\usepackage{multirow}

\newcommand{\code}[1]{\texttt{#1}}

% \newcolumntype{y}[1]{>{\raggedleft\hspace{0pt}}p{#1}}
% \newcolumntype{y}{>{\raggedright }l}
% \newcolumntype{x}[1]{>{\raggedright\hspace{0pt}}p{#1}}

\begin{document}

\title{OmpSs-2@Cluster: Distributed memory execution of nested OpenMP-style tasks}

% \author{
%   Jimmy Aguilar Mena\inst{1} \and %\orcidID{0000-0001-6802-2247} \and
%   Omar Shaaban\inst{1} \and %\ordicID{0000-0003-4410-5317} \and
%   Vicen\c{c} Beltran\inst{1} \and %\orcidID{0000-0002-3580-9630} \and
%   Paul Carpenter\inst{1} \and %\orcidID{0000-0002-9392-0521} \and
%   Eduard Ayguade\inst{1} \and %\orcidID{0000-0002-5146-103X} \and
%   Jesus Labarta\inst{1} %\orcidID{0000-0002-7489-4727}
% }
%\institute{Barcelona Supercomputing Center}

\maketitle

This paper it about features implemented in OmpSs-2@Cluster, so the
first step to reproduce the results is to install Nanos6 and the
source to source compiler Mercurium to create a functional setup.

\section{System, Dependencies and Environment}

We evaluate OmpSs-2@Cluster on MareNostrum\,4. Each node has two
24-core Intel Xeon Platinum 8160 CPUs at 2.10\,GHz, for a total of 48
cores per node.  Each socket has a shared 32\,MB L3 cache.  The HPL
Rmax performance equates to 1.01\,TF per socket.

In general any GNU/Linux system allowing memory over-subscription and
disable address space randomization may work with no problem.


The runtime dependencies are:

\begin{enumerate}
    \item automake, autoconf, libtool, pkg-config and make

    \item C and C++ compiler to build the runtime and the compiler; we have
        tested gcc\,7.2.0 and icpc\,2018.1 and they both work without issues.
        \footnote{To build the tests with Mercurium provides different executables;
        in order to use gcc/g++ (mcc/mcxx) or icc (imc/imcxx).}

    \item boost >= 1.59. In our experiments we use boost\,1.64.0.

    \item hwloc. If you use OpenMPI then it will be a dependency as there are some
        version constrains between OpenMPI and hwloc. In our experiments we use
        version 1.11.8

    \item MPI library. Our communication uses Intel MPI\,2018.4 over 100\,Gb/s Intel
        OmniPath, with an HFI Silicon 100 series PCIe adaptor. To build Nanos6
        with cluster support we require MPI with
        \code{MPI\_THREAD\_MULTIPLE} support.  Intel MPI\,2018.4 can be freely
        downloaded from the Intel website; but the runtime works with OpenMPI
        and MPICH as well. \footnote{The final results may strongly depend of
        the MPI multithread support and optimization; we observed degraded
        performance results when using OpenMPI.}
\end{enumerate}

We need some extra dependencies to build mercurium. They are usually
installed in most of the systems, but in case you have some dependency
issues you may check for: bison, gperf, libsqlite3 and flex.

We have two different set of benchmarks for MPI and
OmpSs-2@Cluster. They use the same libraries, compiler and
dependencies than before but require two extra dependencies.

\begin{enumerate}
    \item cmake>3.10 to build the benchmarks.
    \item BLASS, in our experiments we use MKL\,2018.4. There is a set of benchmarks
        that don't use BLASS kernels, but they are not recommended as some of the
        results may vary from the ones reported in the paper.
\end{enumerate}

Finally to process the output data and generate the graphs we use
python3, with matplotlib, numpy and pandas.

\section{Build and install}

\subsection{Nanos6 build and installation}

The nanos6 basic installation instructions are in:
\url{https://github.com/bsc-pm/nanos6-cluster} which explains the
dependencies and some other features.  OmpSs-2@Cluster has the same
setup and requirements but require only MPI as an extra dependency.

\begin{lstlisting}
export STARTDIR=${PWD}
git clone --depth 1 https://github.com/bsc-pm/nanos6-cluster
cd nanos6-cluster
autoreconf -vif
export NANOS6_HOME=/some/path1
./configure --enable-cluster --enable-execution-workflow --prefix=${NANOS6_HOME} \
            --disable-lint-instrumentation --disable-ctf-instrumentation \
            --disable-graph-instrumentation --disable-stats-instrumentation \
            --disable-extrae-instrumentation --disable-verbose-instrumentation
make -j install
\end{lstlisting}

As usual the options CC=gcc CXX=g++ can be used to specify the C and
C++ compilers --with-boost= to specify BOOST path in case it is not in
the default path.

\subsection{Mercurium build and installation}

\begin{lstlisting}
cd ${STARTDIR}
git clone --depth 1 https://github.com/bsc-pm/mcxx
cd mcxx
autoreconf -vif
export MERCURIUM_HOME=/some/path2
./configure --prefix=${MERCURIUM_HOME} \
            --with-nanos6=${NANOS6_HOME} \
            --enable-ompss-2
make -j install
export PATH=${MERCURIUM_HOME}/bin:${PATH}
\end{lstlisting}

\subsection{Benchmark build and test}

The MPI benchmarks are in different repositories so we need to build
them appart. This is a requirement because cmake does not allow to use
different compilers in the same project; and the MPI benchmarks does
not need to use mercurium.

For the OmpSs-2@Cluster benchmarks:

\begin{lstlisting}
cd ${STARTDIR}
git clone --depth=1 https://github.com/Ergus/nanos-cluster-benchmarks
mkdir nanos-cluster-benchmarks/build
cd nanos-cluster-benchmarks/build
cmake -DCMAKE_BUILD_TYPE=Release ..
make
export NANOS6_CONFIG=${STARTDIR}/nanos6.toml
\end{lstlisting}

For MPI benchmarks:

\begin{lstlisting}
cd ${STARTDIR}
git clone --depth=1 --recursive https://github.com/Ergus/MPI_Benchmarks
mkdir MPI_Benchmarks/build
cd MPI_Benchmarks/build
cmake -DCMAKE_BUILD_TYPE=Release ..
make
\end{lstlisting}

By default cmake will try to use the gcc compiler; to use icc it is
possible to pass the option -DCCOMPILER to cmake.

\begin{lstlisting}
cmake -DCCOMPILER=intel ..
\end{lstlisting}

By default cmake will try to build with the system's Lapack library
following the cmake policy:
\href{https://cmake.org/cmake/help/latest/module/FindLAPACK.html}{FindLAPACK}
\href{https://cmake.org/cmake/help/latest/module/FindBLAS.html}{FindBLAS}

We added a special option \code{-DWITH\_MKL=true} to simplify to force
using Intel MKL (recommended).

If everything was right all the tests may be build now with the
submission scripts in their sub-directories. In an interactive session
you may try: \code{ctest} to check if they can run in your system.

\subsection{Running benchmarks}

The following instructions apply to MPI and OmpSs-2@Cluster
benchmarks.  They are executed in the same way, the command line
arguments are the same, the submission scripts and everything else.

\subsubsection{OmpSs benchmarks (interactively)}
To execute a benchmark in an interactive session you just need to do
(from the build directory):

\begin{lstlisting}
cd ${STARTDIR}/nanos-cluster-benchmarks/build/jacobi_ompss2
mpirun -np $NP ./jacobi_task_fetchall_blas_ompss2 $DIM $BS $ITS
\end{lstlisting}

Where \code{\$NP} is the number of processes to use, \code{\$DIM} the
problem dimension, \code{\$BS} the block size and \code{\$ITS} the
number of iterations to execute.

One example execution and its output may be:
\begin{lstlisting}
> mpirun -np 2 ./jacobi_task_nofetch_blas_ompss2 1024 16 4

# Initializing data
# Starting algorithm
# jacobi tasks FETCHTASK=0
# Finished algorithm...
Executable: "./jacobi_task_nofetch_blas_ompss2"
Rows: 1024
Tasksize: 16
Iterations: 4
Print: 0
worldsize: 2
cpu_count: 24
namespace_enabled: 1
nanos6_version: "2.5.1 2022-05-08 21:40:06 +0200 2a61a546"
Total_time: 9.47129e+07
Algorithm_time: 7.96941e+07
\end{lstlisting}

Our repository provides a script \code{interactive\_dim.sh} to execute
the benchmarks interactively. The script is automatically copied into
the executable directories and its use is very simple:

\begin{lstlisting}
./interactive_dim.sh -N 1,2 -R 1 -D 32768 -B 64 -I 1 exe1 exe2 exe3
\end{lstlisting}

This will execute mpirun with -n \code{1} and \code{2} nodes (-N) for
the executable files: \code{exe1}, \code{exe2} and
\code{exe3}. Repeating the execution \code{3} times (-R). With a
problem dimension of \code{32768} (-D), block size \code{64} (-B) and
for \code{5} iterations (-I).


\subsubsection{OmpSs benchmarks (Slurm cluster submit)}

The repository also provide the scripts to submit the jobs in a
cluster with Slurm in case your system has such a setup.

The submit scripts are inside every just created benchmark directory
named \code{submiter\_dim.sh}. The submit scripts multiple command line
options that you can read directly in the file:

For example:

\begin{lstlisting}
./submiter_dim.sh -N 1,2 -R 3 -C 24 -D 32768 -B 32,64 -I 5 -o results exe1 exe2 exe3
\end{lstlisting}

This will submit 3 jobs with sbatch for \code{1} and \code{2} nodes
(-N) respectively. That will run the executable files: \code{exe1},
\code{exe2} and \code{exe3} with srun \code{3} times each (-R) in a
loop with \code{24} cores per process (-C) and will create a directory
\code{results} (-o) to save all the outputs. With a dimension of
\code{32768} (-D), block sizes of \code{32} and \code{64} (-B) and for
\code{5} iterations (-I).

The script also creates a file inside results named
\code{results/submit.log} with information about the submitted jobs in
case you need to re-check them latter.

\section{Step by step instructions}

To partially reproduce some of the results in the paper the execution
may require to use a big setup (up to 16 nodes and 48 cores per node
in two numa nodes). You can reduce the problem size (-D), the number
of iterations (-I) and the time the experiments are repeated (-R) to
reduce the execution time. This same commands can be used either with
the interactive or the submit scripts.

\begin{lstlisting}
cd ${STARTDIR}/nanos-cluster-benchmarks/build/matmul_matvec_ompss2

cp ${STARTDIR}/MPI_Benchmarks/build/matmul_matvec_mpi/matvec_parallelfor_blas_mpi .
./interactive_dim.sh -N 1,2 -R 10 -D 65536 -B 256 -I 1 \
	matvec_strong_flat_task_node_blas_ompss2 \
	matvec_weak_fetchall_task_node_blas_ompss2 \
	matvec_parallelfor_blas_mpi | tee ${STARTDIR}/output_matvec.txt

cp ${STARTDIR}/MPI_Benchmarks/build/matmul_matvec_mpi/matmul_parallelfor_blas_mpi .
./interactive_dim.sh -N 1,2 -R 10 -D 16384 -B 16 -I 1 \
	matmul_strong_nested_task_node_blas_ompss2 \
	matmul_weak_fetchall_task_node_blas_ompss2 \
	matmul_parallelfor_blas_mpi | tee ${STARTDIR}/output_matmul.txt

cd ${STARTDIR}/nanos-cluster-benchmarks/build/jacobi_ompss2
cp ${STARTDIR}/MPI_Benchmarks/build/jacobi_mpi/jacobi_parallelfor_nop2p_blas_mpi .
./interactive_dim.sh -N 1,2 -R 10 -D 65536 -B 256 -I 1 \
	jacobi_task_fetchall_blas_ompss2 \
	jacobi_taskfor_blas_ompss2 \
	jacobi_parallelfor_nop2p_blas_mpi | tee ${STARTDIR}/output_jacobi.txt

cd ${STARTDIR}/nanos-cluster-benchmarks/build/cholesky_fare_ompss2
cp ${STARTDIR}/MPI_Benchmarks/build/cholesky_mpi/cholesky_omp_mpi .
./interactive_dim.sh -N 1,2 -R 10 -D 65536 -B 512 -I 1 \
	cholesky_fare_strong_ompss2 \
	cholesky_fare_taskfor_ompss2 \
	cholesky_omp_mpi | tee ${STARTDIR}/output_cholesky.txt

\end{lstlisting}

\end{document}

%%% Local Variables:
%%% mode: latex
%%% TeX-master: t
%%% End:
